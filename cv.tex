\documentclass[11pt,a4paper]{moderncv}
\moderncvtheme[blue]{classic}
\usepackage[utf8]{inputenc}
\usepackage[top=1.1cm, bottom=1.1cm, left=2cm, right=2cm]{geometry}
% Largeur de la colonne pour les dates
\setlength{\hintscolumnwidth}{2.5cm}

\photo{photo.jpg}
\firstname{Yannick}
\familyname{Utard}
\title{Étudiant en informatique}              
\address{31 grand'rue}{68320 Wickerschwihr}    
\email{yannickutard@gmail.com}                      
\homepage{github.com/utay}
\mobile{06 78 06 81 35} 
\extrainfo{19 ans}
\begin{document}
\maketitle

\section{Formations}
\cventry{2014 à ce jour}{EPITA}{École d'ingénieur en informatique}{Paris}{}{}
\cventry{2014}{Baccalauréat}{mention Assez Bien}{}{}{Série scientifique, option Informatique et Sciences du numérique}

\section{Expériences}
\subsection{Emplois saisonniers}
\cventry{Juillet 2014}{Colmarienne des Eaux}{Colmar}{}{}{Assistant informatique}
\cventry{Juillet 2013}{Giamberini et Guy}{Turckheim}{}{}{Ouvrier en entreprise de paysagisme et de génie civil }

\subsection{Stages de découverte}
\cventry{Février 2011}{Exalead}{Paris}{}{}{Découverte du fonctionnement d’un moteur de recherche}
\cventry{}{Sucrerie}{Erstein}{}{}{Découverte de divers postes}

\subsection{Projets}
\cventry{2011 - 2015}{Développement d’un jeu vidéo}{projet personnel}{}{}{Réalisation d'un jeu vidéo en 4 mois, sous Unity 3D, en C\#}
\cventry{}{Développement d’un logiciel de reconnaissance faciale}{projet étudiant}{}{}{Réalisation d'un logiciel de reconnaissance faciale avec la méthode de Viola \& Jones, en C}
\cventry{}{Réalisation de plusieurs sites web}{milieu scolaire et projet personnel}{}{}{Réalisation de plusieurs sites web en HTML/CSS/PHP dont un site de "mini-entreprise" et un jeu en ligne}

\section{Compétences}
\cvitem{Langages}{\textbf{C}, \textbf{C\#}, \textbf{Ruby}, Java, Caml, Bash,  \LaTeX}
\cvitem{Web}{HTML, CSS, PHP}
\cvitem{IDE}{\textbf{Emacs}, Visual Studio, Unity 3D}
\cvitem{Versionning}{Git}
\cvitem{Réseaux}{Connexion, installation}
\cvitem{Environnements}{Windows, Linux}
\cvitem{Bureautique}{Microsoft Office, GIMP}
\cvitem{Anglais}{lu, écrit, parlé - \textbf{Toeic 800}}{}
\cvitem{Allemand}{lu, écrit}{}

\section{Centres d'intérêts et autres}
\cvitem{Sports}{Musculation, cyclisme, football}
\cvitem{}{Titulaire du permis B}


\end{document}
